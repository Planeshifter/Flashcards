\documentclass[avery5371,grid]{flashcards}

\cardfrontstyle{headings}
\cardfrontfoot{Advanced Probability}

\usepackage{amssymb}
\usepackage{amsmath}
\usepackage{amsfonts}
\usepackage{mathrsfs}
\usepackage{datetime}
\usepackage{bbm}

\begin{document}

\begin{flashcard}[Definition]{Increasing Sequence of Sets}
Let $A_1,A_2,\ldots$ be subsets of $\Omega$. If $A_1 \subset A_2 \subset \ldots$ and 
$\cup_{n=1}^\infty = A$, $A_n$ is said to form an increasing sequence of sets with limit A.
\end{flashcard}

\begin{flashcard}[Definition]{Decreasing Sequence of Sets}
Let $A_1,A_2,\ldots$ be subsets of $\Omega$. If $A_1 \supset A_2 \supset \ldots$ and 
$\cap_{n=1}^\infty = A$, $A_n$ is said to form a decreasing sequence of sets with limit A.
\end{flashcard}


\begin{flashcard}[Theorem]{De Morgan Laws}
Let $A_1,A_2,\ldots$ be subsets of $\Omega$. We have
\[
\left(\cup_n A_n\right)^c = \cap_n A_n^c 
\]
and
\[
\left( \cap_n A_n \right)^c = \cup_n A_n^c
\]
\end{flashcard}


\begin{flashcard}[Definition]{Fields and $\sigma$-Fields}
\scriptsize
Let $\Omega$ be a set. A collection $\mathcal{F}$ of subsets of $\Omega$ is called a 
field if it is closed under complementation and finite union:
\begin{itemize}
 \item $\Omega \in \mathcal{F}$
 \item for each $A \in \mathcal{F}$, $A^C \in \mathcal F$, 
 \item for all $A_1,A_2 \in \mathcal{F}$, $A_1 \cup A_2 \in \mathcal{F}$ 
\end{itemize}
From this, it follows that $\mathcal{F}$ is closed under finite intersection:
\[
\cap_{i=1}^n A_i = \left( \cup_{i=1}^n A_i^c \right)^c \in \mathcal{F}
\]
A field is called a $\sigma$-field if it also satisfies the condition that for every
sequence $\{A_k\}_{k=1}^\infty \in \mathcal F$, we have that  $\cup_{k=1}^\infty A_k \in \mathcal{F}$
\end{flashcard}

\begin{flashcard}[Definition]{Simple Function}
 A measurable function that takes at most finitely many values is called a simple
 function.
\end{flashcard}

\begin{flashcard}[Definition]{Canonical Representation of Simple Function}
Let $f$ be a simple function whose distinct values are $a_1, \ldots, a_n$ and let
$A_i = \left\{ \omega : f\left( \omega \right) = a_i \right\}$. Then 
$f = \sum_{i=1}^n a_i \mathbbm{I}_{A_i}$ is called the canonical representation of
$f$. 
\end{flashcard}

\begin{flashcard}[Lemma]{Monotone Approximation}
 Let $f$ be a nonnegative measurable extended real-valued function from $\Omega$. Then there exists
 a sequence $\left\{ f_n \right\}_{n=1}^\infty$ of nonnegative finite simple functions
 such that $f_n \le f$ for all $n$ and $\lim_{n \to \infty} f_n\left( \omega \right) = f\left( \omega \right) \forall \omega$.
\end{flashcard}

\begin{flashcard}[Definition]{Splitting Measurable Functions}
Let $f$ be a real-valued function. The positive part $f^+$ of $f$ is defined as
$f^+\left( \omega \right) = \max \left\{ f\left( \omega \right), 0\right\}$. The negative part
$f^-$ of $f$ is $f^-\left( \omega \right)=-min\left\{ f\left( \omega \right), 0 \right\}$. We have
$f = f^+ - f^-$.
\end{flashcard}

\begin{flashcard}[Definition]{Integral of Simple Functions}
 Let $f : \Omega \to \bar{\mathbb{R}}^{+0}$ be a simple function with canonical 
 representation $f\left( \omega \right) = \sum_{i=1}^n \mathbbm{I}_{A_i}\left( \omega \right)$.
 The integral of $f$ with respect to $\mu$ is defined to be $\sum_{i=1}^n a_i \mu\left( A_i \right)$. The integral is denoted
 variously as $\int f d\mu$, $\int f\left( \omega \right) \mu\left( d\omega \right)$ or
 $\int f\left( \omega \right) d\mu\left( \omega \right)$.
\end{flashcard}

\begin{flashcard}[Proposition]{Integral for two functions $f \le g$}
If $f \le g$ and both are nonnegative and simple, then 
\[
\int f d\mu \le \int g d\mu.
\]
\end{flashcard}

\begin{flashcard}[Definition]{Integrable Function}
We say that $f$ is integrable with respect to $\mu$ if $\int f d\mu$ is finite. 
\end{flashcard}

\begin{flashcard}[Definition]{Integral for General Nonnegative Functions}
 For nonnegative measurable $f$, we define the integral of $f$ with respect to $\mu$ by
 \[
 \int f d\mu = \sup_{\text{nonnegative finite simple } g \le f} \int g d\mu 
 \]
\end{flashcard}

\begin{flashcard}[Definition]{Integral of General Measurable Functions}
Let $f$ be measurable. If either $f^+$ or$f^-$ is integrable with respect to $\mu$, we 
define the integral of $f$ with respect to $\mu$ to be 
\[
\int f^+ d\mu - \int f^- d\mu,
\]
 otherwise the integral does not exist.
\end{flashcard}

\begin{flashcard}[Definition]{Integration over a set}
 If $A \in \mathcal{F}$, we define 
 \[
 \int_A f d\mu = \int \mathbbm{I}_A f d\mu.
 \]
\end{flashcard}

\begin{flashcard}[Proposition]{Monotonicity of Integral}
 If $f \le g$ and both integrals are defined, then $\int f d\mu \le \int g d\mu$.
\end{flashcard}

\begin{flashcard}[Definition]{Almost sure / almost everywhere}
 Suppose that some statement about elements of $\Omega$ holds for all $\omega \in A^C$,
 where $\mu\left( A \right) = 0$. Then we say that the statement holds almost everywhere,
 denotes as $a.e. \; [\mu]$. If $P$ is a probability, then almost everywhere is often replaced
 by almost surely, denoted $a.s. \; [P]$.
\end{flashcard}

\begin{flashcard}[Theorem]{Additivity}
 \[
 \int \left( f +g \right) d\mu = \int f d\mu + \int g d\mu,
 \]
 whenever at least two of them are finite. 
\end{flashcard}

\begin{flashcard}[Theorem]{Change of Variable}
Let $\left( \Omega, \mathcal{F}, \mu \right)$ be a measure space and let $\left( S, \mathcal{A} \right)$
be a measurable space. Let $f : \Omega \mapsto S$ be a measurable function. Let
$\nu$ be the measure induced on $\left( S,\mathcal{A} \right)$ by $f$ from
$\mu$. Let $g : S \mapsto \mathbb{R}$ be $A/\mathcal{B}^1$ measurable. Then
\[
\int g d\nu = \int g(f) d\mu,
\]
if either integral exists.
\end{flashcard}

\begin{flashcard}[Theorem]{Relationship to Riemann Integral}
Let $f$ be a continuous function on a closed bounded interval $\left[ a,b \right]$.
Let $\mu$ be Lebesgue measure. Then the Riemann integral $\int_a^b f(x) dx$ equals
$\int_{\left[ a,b \right]} f d\mu$.
\end{flashcard}

\begin{flashcard}[Definition]{Expectation and Variance of Random Variables}
 If $P$ is a probability and $X$ is a random variable, then
 $\int X dP$ is called the mean of $X$, expected value of $X$,  or expectation of
 $X$, and denoted by $\mathbb{E}\left( X \right)$. If $\mathbb{E} = \mu$ is finite, 
 then the variance of $X$ is $\operatorname{Var}\left( X \right) = \mathbb{E} \left[ \left( X - \mu \right)^2 \right]$
\end{flashcard}




\end{document}
