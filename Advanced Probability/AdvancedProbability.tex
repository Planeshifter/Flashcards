\documentclass[avery5371,grid]{flashcards}

\cardfrontstyle{headings}
\cardfrontfoot{Advanced Probability}

\usepackage{amssymb}
\usepackage{amsmath}
\usepackage{amsfonts}
\usepackage{amsthm}
\usepackage{mathrsfs}
\usepackage{datetime}
\usepackage{bbm}

\begin{document}

\begin{flashcard}[Definition]{Increasing Sequence of Sets}
Let $A_1,A_2,\ldots$ be subsets of $\Omega$. If $A_1 \subset A_2 \subset \ldots$ and 
$\cup_{n=1}^\infty = A$, $A_n$ is said to form an increasing sequence of sets with limit A.
\end{flashcard}

\begin{flashcard}[Definition]{Decreasing Sequence of Sets}
Let $A_1,A_2,\ldots$ be subsets of $\Omega$. If $A_1 \supset A_2 \supset \ldots$ and 
$\cap_{n=1}^\infty = A$, $A_n$ is said to form a decreasing sequence of sets with limit A.
\end{flashcard}


\begin{flashcard}[Theorem]{De Morgan Laws}
Let $A_1,A_2,\ldots$ be subsets of $\Omega$. We have
\[
\left(\cup_n A_n\right)^c = \cap_n A_n^c 
\]
and
\[
\left( \cap_n A_n \right)^c = \cup_n A_n^c
\]
\end{flashcard}


\begin{flashcard}[Definition]{Fields and $\sigma$-Fields}
\scriptsize
Let $\Omega$ be a set. A collection $\mathcal{F}$ of subsets of $\Omega$ is called a 
field if it is closed under complementation and finite union:
\begin{itemize}
 \item $\Omega \in \mathcal{F}$
 \item for each $A \in \mathcal{F}$, $A^C \in \mathcal F$, 
 \item for all $A_1,A_2 \in \mathcal{F}$, $A_1 \cup A_2 \in \mathcal{F}$ 
\end{itemize}
From this, it follows that $\mathcal{F}$ is closed under finite intersection:
\[
\cap_{i=1}^n A_i = \left( \cup_{i=1}^n A_i^c \right)^c \in \mathcal{F}
\]
A field is called a $\sigma$-field if it also satisfies the condition that for every
sequence $\{A_k\}_{k=1}^\infty \in \mathcal F$, we have that  $\cup_{k=1}^\infty A_k \in \mathcal{F}$
\end{flashcard}

\begin{flashcard}[Definition]{Measurable Space and Measurable Sets}
A set $\Omega$ together with $\sigma$-algebra $\mathcal{F}$ is called a measurable space. The elements 
of $\mathcal{F}$ are called measurable sets.
\end{flashcard}

\begin{flashcard}[Definition]{$\sigma$-field generated by $\mathcal{A}$}
 Let $\mathcal{A}$ be a collection of subsets of $\Omega$. We denote with $\sigma\left( \mathcal{A} \right)$
 the smallest $\sigma$-field containing $\mathcal{A}$, which is called the $\sigma$-field generated by $\mathcal{A}$. 
\end{flashcard}

\begin{flashcard}[Theorem]{Existence of $\sigma\left( \mathcal{A} \right)$}
The proof has three main steps:
\begin{itemize}
 \item Define all $\sigma$-fields containing $\mathcal{A}$ as $\mathcal{F}_i, i \in \mathcal{I}$,
 where $\mathcal{I}$ is some index set. Note that one such $\sigma$-field always exists since 
 $\mathcal{A} \subset 2^\Omega$.
 \item Show that the intersection of all $\mathcal{F}_i$ is again 
 a $\sigma$-field 
 \item Finally, show that $\cap_{i \in \mathcal{I}} \mathcal{F}_i$ is moreover the smallest 
 $\sigma$-field possible.
\end{itemize}

\end{flashcard}

\begin{flashcard}[Definition]{Extended Reals}
 The extended real numbers are $\bar{\mathbb{R}} = \mathbb{R} \cup \{-\infty,\infty \}$. The positive extended realls
 are $\bar{\mathbb{R}}^+ = \left( 0,\infty  \right]$ and the nonnegative
 extended reals are $\bar{\mathbb{R}}^{+0} = \left[ 0,\infty  \right]$.
\end{flashcard}

\begin{flashcard}[Definition]{Rules of Arithmetic in $\bar{\mathbb{R}}$}
Let $c \in \mathbb{R}$.
\begin{itemize}
 \item $c + \infty = \infty$ and $c - \infty = -\infty$
 \item $\infty + \infty = \infty$ and $-\infty - \infty = - \infty$ (however,
 $\infty - \infty$ is NOT defined!)
 \item $0 \cdot \infty = 0$ and $\frac{c}{\infty} = \frac{c}{-\infty} = 0$ (however, $\frac{\infty}{\infty}$
 is not defined
 \item $c \times \infty = \infty$ if $c > 0 $ and $c \times \infty = - \infty$ if $c < 0$. 
\end{itemize}
 
\end{flashcard}

\begin{flashcard}[Definition]{Measures}
\scriptsize
 Let $\left( \Omega,\mathcal{F} \right)$ be a measurable space. Let $\mu : \mathcal{F} \to \bar{\mathbb{R}}^{+0}$
 satisfy:
 \begin{itemize}
  \item $\mu\left( \emptyset \right) = 0$
  \item For any sequence of mutually disjoint sets $\left\{ A_n \right\}_{n=1}^\infty$ of $\mathcal{F}$ (i.e. $A_i \cap A_j = \emptyset$ for
  $i \ne j$), we have
  \[
  \mu\left( \cup A_{n=1}^\infty  \right) = \sum_{n=1}^\infty \mu\left( A_n \right)
  \]
 \end{itemize}
We call $\left( \Omega, \mathcal{F}, \mathcal{\mu} \right)$ a measure space. If $\mu\left( \Omega \right) = 1$,
it is a probability space and we usually write $P$ instead of $\mu$.
\end{flashcard}

\begin{flashcard}[Definition]{Probability Measure}
 A measure $P$ which satisfies $P\left( \Omega \right) = 1$ is called a probability measure. Then,
 the measure space $\left( \Omega,\mathcal{F},P \right)$ is called a probability space,
 and the sets of $\mathcal{F}$ are called events.
\end{flashcard}

\begin{flashcard}[Definition]{Finite and $\sigma$-finite Measures}
\scriptsize
A measure $\mu$ is finite if for all $A \in \mathcal{F}$, we have $\mu\left( A \right) < \infty$. It is called
$\sigma$-finite if there exists a sequence $\left\{ A_n \right\}_{n=1}^\infty$ such that $\mu\left( A_n \right) < \infty \, \forall n$
and $\cup_{n=1}^\infty A_n = \Omega$. 

Observe that a finite measure is always 
$\sigma$-finite, but the reverse is not true: Take for example the counting measure and let $\mathcal{F}$ be the $\sigma$-field generated by the natural 
numbers. Then clearly $\mu\left( \Omega \right) = \mu\left( \mathbb{N} \right) = \infty$, so 
the measure is not finite. However, it is $\sigma$-finite since we can define $A_n = \left\{  n \right\}$ with $\mu\left( A_n \right) =1$
and $\cup_{n=1}^\infty A_n = \mathbb{N}$.
\end{flashcard}

\begin{flashcard}[Theorem]{Countable Subadditivity of Measure}
 For an arbitrary sequence $\left\{ A_n \right\}_{n=1}^\infty$, we have
 \[
 \mu\left( \cup_{n=1}^\infty A_n \right) \le \sum_{i=1}^n \mu\left( A_n \right).
 \]
 \scriptsize
 \begin{proof}
  Let $B_n = A_n - \left( \cup_{i < n} A_i \right)$. Then $\left\{ B_n \right\}$ forms
  a disjoint sequence of sets which satisfies $\cup_{n=1}^\infty B_n = \cup_{n=1}^\infty A_n$. We thus have
  $ \mu\left( \cup_{n=1}^\infty A_n \right) =  \mu\left( \cup_{n=1}^\infty B_n \right) = \sum_{i=1}^n \mu\left( B_n \right)$.
  Since $\mu\left( B_n \right) \le \mu\left( A_n \right) \forall n$, the result follows.
 \end{proof}

\end{flashcard}

\begin{flashcard}[Theorem]{Further Properties of Measure}
 \begin{itemize}
  \item Linearity: If $\mu_1, \mu_2, \ldots$ are measures on $\left( \Omega,\mathcal{F} \right)$,
  then $\mu = \sum_j a_j \mu_j $ is also a measure on $\left( \Omega,\mathcal{F} \right)$.
  \item If $\mu(A_n) = 0$ for all $A_n$, then $\mu(\cup_{n=1}^\infty) = 0$.
  \item If $\mu(A_n) = 1$ for all $A_n$, then $\mu\left( \cap_{n=1}^\infty A_n \right) = 1$.
 \end{itemize}

\end{flashcard}

\begin{flashcard}[Definition]{Monotone Sequences of Sets}
 For a measure space $\left( \Omega,\mathcal{F},\mu \right)$,
 a sequence $\left\{ A_n \right\}_{n=1}^\infty$ of elements in $\mathcal{F}$
 is called monotone increasing if $A_n \subseteq A_{n+1} \forall n$. If on the other hand
 $A_n \supseteq A_{n+1}$, it is a monotinically decreasing sequence of sets.
\end{flashcard}

\begin{flashcard}[Definition]{Measurable Function}
 Let $\left( \Omega,\mathcal{F} \right)$ and $\left( S,\mathcal{A} \right)$
 be measurable spaces. Then a function $f$ which maps from $\Omega$ to $S$ is called
 $\mathcal{F} / \mathcal{A}$ measurable if $f^{-1}\left( A \right) \in \mathcal{F} \; \forall A \in \mathcal{A}$,
 where the inverse image or pre-iamge $f^{-1}\left( A \right)$ is defined as
 $f^{-1}\left( A \right) = \left\{ \omega \in \Omega : f\left( \omega \right) \in \mathcal{A} \right\}$. For sake of 
 brevity, we might just say that $f$ is measurable in this case.
\end{flashcard}

\begin{flashcard}[Theorem]{Properties of Measurable Functions}
 Let $\left( \Omega,\mathcal{F} \right)$, $\left( S,\mathcal{A} \right)$
 and $\left( T,\mathcal{B} \right)$ be measurable spaces. Then
 \begin{itemize}
  \item If $f : \Omega \mapsto \bar{\mathbb{R}}$ and $c$ is some constant, then
  $cf$ is measurable.
  \item if $f : \Omega \mapsto S$ and $g : S \mapsto T$, then the composition $g \circ f = g(f) : \Omega \mapsto T$ is
  measurable. 
  \item If $f$ and $f$ are measurable real-valued functions, so are $f+g$ and $fg$.
 \end{itemize}
\end{flashcard}

\begin{flashcard}[Theorem]{Properties of Sequence of Measurable Functions}
 Let $f_n$ be a sequence of measurable functions which satisfies $f_n : \Omega \mapsto \mathbb{R}$
 for all $n$. Then the following are measurable:
 \begin{itemize}
  \item $\limsup_n f_n$ and $\liminf_n f_n$
  \item $\left\{ \omega : \lim_{n \to \infty} f_n\left( \omega \right) \text{exists}\right\}$
  \item $f = \begin{cases}
              \lim_{n \to \infty} f_n & \text{where the limit exists} \\
              0 & \text{elsewhere}
             \end{cases}
$
 \end{itemize}

\end{flashcard}

\begin{flashcard}[Definition]{Simple Function}
 A measurable function that takes at most finitely many values is called a simple
 function.
\end{flashcard}

\begin{flashcard}[Definition]{Canonical Representation of Simple Function}
Let $f$ be a simple function whose distinct values are $a_1, \ldots, a_n$ and let
$A_i = \left\{ \omega : f\left( \omega \right) = a_i \right\}$. Then 
$f = \sum_{i=1}^n a_i \mathbbm{I}_{A_i}$ is called the canonical representation of
$f$. 
\end{flashcard}

\begin{flashcard}[Lemma]{Monotone Approximation}
 Let $f$ be a nonnegative measurable extended real-valued function from $\Omega$. Then there exists
 a sequence $\left\{ f_n \right\}_{n=1}^\infty$ of nonnegative finite simple functions
 such that $f_n \le f$ for all $n$ and $\lim_{n \to \infty} f_n\left( \omega \right) = f\left( \omega \right) \forall \omega$.
\end{flashcard}

\begin{flashcard}[Definition]{Splitting Measurable Functions}
Let $f$ be a real-valued function. The positive part $f^+$ of $f$ is defined as
$f^+\left( \omega \right) = \max \left\{ f\left( \omega \right), 0\right\}$. The negative part
$f^-$ of $f$ is $f^-\left( \omega \right)=-\min\left\{ f\left( \omega \right), 0 \right\}$. We have
$f = f^+ - f^-$ and $|f| = f^+ + f^-$.
\end{flashcard}

\begin{flashcard}[Definition]{Integral of Simple Functions}
 Let $f : \Omega \to \bar{\mathbb{R}}^{+0}$ be a simple function with canonical 
 representation $f\left( \omega \right) = \sum_{i=1}^n \mathbbm{I}_{A_i}\left( \omega \right)$.
 The integral of $f$ with respect to $\mu$ is defined to be $\sum_{i=1}^n a_i \mu\left( A_i \right)$. The integral is denoted
 variously as $\int f d\mu$, $\int f\left( \omega \right) \mu\left( d\omega \right)$ or
 $\int f\left( \omega \right) d\mu\left( \omega \right)$.
\end{flashcard}

\begin{flashcard}[Proposition]{Integral for two functions $f \le g$}
If $f \le g$ and both are nonnegative and simple, then 
\[
\int f d\mu \le \int g d\mu.
\]
\end{flashcard}

\begin{flashcard}[Definition]{Integrable Function}
We say that $f$ is integrable with respect to $\mu$ if $\int f d\mu$ is finite. 
\end{flashcard}

\begin{flashcard}[Definition]{Integral for General Nonnegative Functions}
 For nonnegative measurable $f$, we define the integral of $f$ with respect to $\mu$ by
 \[
 \int f d\mu = \sup_{\text{nonnegative finite simple } g \le f} \int g d\mu 
 \]
\end{flashcard}

\begin{flashcard}[Definition]{Integral of General Measurable Functions}
Let $f$ be measurable. If either $f^+$ or$f^-$ is integrable with respect to $\mu$, we 
define the integral of $f$ with respect to $\mu$ to be 
\[
\int f^+ d\mu - \int f^- d\mu,
\]
 otherwise the integral does not exist.
\end{flashcard}

\begin{flashcard}[Definition]{Integration over a set}
 If $A \in \mathcal{F}$, we define 
 \[
 \int_A f d\mu = \int \mathbbm{I}_A f d\mu.
 \]
\end{flashcard}

\begin{flashcard}[Proposition]{Monotonicity of Integral}
 If $f \le g$ and both integrals are defined, then $\int f d\mu \le \int g d\mu$.
\end{flashcard}

\begin{flashcard}[Definition]{Almost sure / almost everywhere}
 Suppose that some statement about elements of $\Omega$ holds for all $\omega \in A^C$,
 where $\mu\left( A \right) = 0$. Then we say that the statement holds almost everywhere,
 denotes as $a.e. \; [\mu]$. If $P$ is a probability, then almost everywhere is often replaced
 by almost surely, denoted $a.s. \; [P]$.
\end{flashcard}

\begin{flashcard}[Theorem]{Additivity}
 \[
 \int \left( f +g \right) d\mu = \int f d\mu + \int g d\mu,
 \]
 whenever at least two of them are finite. 
\end{flashcard}

\begin{flashcard}[Theorem]{Change of Variable}
Let $\left( \Omega, \mathcal{F}, \mu \right)$ be a measure space and let $\left( S, \mathcal{A} \right)$
be a measurable space. Let $f : \Omega \mapsto S$ be a measurable function. Let
$\nu$ be the measure induced on $\left( S,\mathcal{A} \right)$ by $f$ from
$\mu$. Let $g : S \mapsto \mathbb{R}$ be $A/\mathcal{B}^1$ measurable. Then
\[
\int g d\nu = \int g(f) d\mu,
\]
if either integral exists.
\end{flashcard}

\begin{flashcard}[Theorem]{Relationship to Riemann Integral}
Let $f$ be a continuous function on a closed bounded interval $\left[ a,b \right]$.
Let $\mu$ be Lebesgue measure. Then the Riemann integral $\int_a^b f(x) dx$ equals
$\int_{\left[ a,b \right]} f d\mu$.
\end{flashcard}

\begin{flashcard}[Definition]{Expectation and Variance of Random Variables}
 If $P$ is a probability and $X$ is a random variable, then
 $\int X dP$ is called the mean of $X$, expected value of $X$,  or expectation of
 $X$, and denoted by $\mathbb{E}\left( X \right)$. If $\mathbb{E} = \mu$ is finite, 
 then the variance of $X$ is $\operatorname{Var}\left( X \right) = \mathbb{E} \left[ \left( X - \mu \right)^2 \right]$
\end{flashcard}

\begin{flashcard}[Theorem]{Fatou's Lemma}
Let $\{ f_n \}_{n=1}^\infty$ be a sequence of nonnegative measurable functions. Then
\[
\int \liminf_n f_n d\mu \le \liminf_n \int f_n d\mu.
\]
\end{flashcard}

\begin{flashcard}[Theorem]{Monotone convergence theorem}
 Let $\{ f_n \}_{n=1}^\infty$ be a sequence of measurable nonnegative functions, and let $f$
be a measurable function such that $f_n \le f$ and $\lim_{n \to \infty} f_n = f$. Then
\[
\lim_{n \to \infty} \int f_n d\mu = \int f d\mu
\]
\end{flashcard}

\begin{flashcard}[Theorem]{Linearity of Integral}
 If $\int f d\mu$ and $\int g d\mu$ are defined and they are not both infinite and of 
 opposite signs, then $\int \left[ f + g \right] d\mu = \int f d\mu + \int g d\mu $. 
\end{flashcard}

\begin{flashcard}[Lemma]{Change of Variables}
  Let $\left( \Omega,\mathcal{F},\mu \right)$ be a measure space and $\left( S,\mathcal{A} \right)$
  be a measurable space. Let $f: \Omega \mapsto S$ be a measurable funciton. Let $\nu$ 
  be the measure induced on $\left( S, \mathcal{A} \right)$ by $f$ from $\mu$. Let $g : S \mapsto \mathbb{R}$
  be $\mathcal{A}/\mathcal{B}^1$ measurable. Then
  \[
  \int g d\nu = \int g(f) d\mu,
  \]
  if either integral exists.
\end{flashcard}

\begin{flashcard}[Corollary]{Law of the unconsious statistician}
  If $X : \Omega \mapsto S$ is a random quantity with distribution $\mu_X$ and if 
  $f : S \mapsto \mathbb{R}$ is measurable, then $\mathbb{E} \left[ f\left( X \right) \right] = \int f d\mu_X$.  
\end{flashcard}

\begin{flashcard}[Theorem]{Density Functions}
  Let $\left( \Omega,\mathcal{F},\mu \right)$ be a measure space, and let $f : \Omega \mapsto \bar{\mathbb{R}}^{+0}$
  be measurable. Then $\nu\left( A \right) = \int_A f d\mu$ is a measure on $\left( \Omega,\mathcal{F} \right)$. The function $f$ is
  called the density of $\nu$ with respect to $\mu$. Integrals with respect to
  $\nu$ can be computed as $\int g d\nu = \int f g d\mu$, if either exists. 
\end{flashcard}

\begin{flashcard}[Theorem]{Dominated convergence theorem}
  Let $\{ _n \}_{n=1}^\infty$ be a sequence of measurable functions, and let 
  $f$ and $g$ be measurabe functions such that $f_n \to f \text{ a.e. }[\mu]$,
  $|f_n|\le g \text{ a.e. } [\mu]$ and $\int g d\mu < \infty$. Then
  \[
  \lim_{n \to \infty} \int f_n d\mu = \int f d\mu.
  \]
\end{flashcard}

\begin{flashcard}[Proposition]{Proposition 18}
  Let $\{ f_n \}_{n=1}^\infty$, $\{ g_n \}_{n=1}^\infty$ be sequences of measurable functions
  such that $|f_n| \le g_n, \text{ a.e. } [\mu]$. Let $f$ and $g$ be measurable functions
  such that $\lim_{n \to \infty} f_m =f$ and $\lim_{n \to \infty} g_n = n$,
  $\text{a.e}[\mu]$. Suppose that $\lim_{n \to \infty}=\int g d\mu < \infty$. Then,
  $\lim_{n \to \infty} \int f_n d\mu = \int f d\mu$.
\end{flashcard}

\begin{flashcard}[Definition]{Uniform Integrability}
  A sequence of integrable functions $\{ f_n \}_{n=1}^\infty$ is  uniformly 
  integrable (with respect to  $\mu$) if 
  $\lim_{c \to \infty}\sup_n  \int_{\{ \omega : |f_n\left( \omega \right)| > c\}} |f_n| d\mu = 0$.
\end{flashcard}

\begin{flashcard}[Theorem]{Properties of Integrals}
\scriptsize
  Let $\left( \Omega,\mathcal{F},\mu \right)$ be a measure space. Let $f$ and
  $g$ be measurable extended real-valued functions.
  \begin{itemize}
   \item If $f$ is nonnegative and $\mu\left( \left\{ \omega : f\left( \omega \right) > 0 \right\} \right)$,
   then $\int f d\mu > 0$.
   \item If $f$ and $g$ are integrable and if $\int_A f d\mu = \int_A g d\mu$ 
   for all $A \in \mathcal{F}$, then $f = g \text{ a.e. }[\mu]$.
   \item If $\mu$ is $\sigma$-finite and if $\int_A f d\mu = \int_A g d\mu$ for all
   $A \in \mathcal{F}$, then $f = g \text{ a.e. }[\mu]$.
   \item Let $\Pi$ be a $\pi$-system that generates $\mathcal{F}$. Suppose that $\Omega$ is a finite or 
   countable union of elements of $\Pi$. If $f$ and $g$ are integrable and if
   $\int_A f d\mu = \int_A g d\mu$ for all $A \in \Pi$, then $f = g \text{ a.e } [\mu]$.
  \end{itemize}
\end{flashcard}

\begin{flashcard}[Corollary]{Corollary 22}
  If $\mu$ is $\sigma$-finite and $\nu$ is related to $\mu$ as in Theorem 10, then the density
  of $\nu $ with respect to $\mu$ is unique, a.e. $[\mu]$.
\end{flashcard}

\begin{flashcard}[Theorem]{Theorem 23}
  Let $\left( \Omega,\mathcal{F},\mu \right)$ be a measure space. Then $\mu$ is $\sigma$-finite if and only if
  there exists a strictly positive integrable function.
\end{flashcard}

\end{document}
